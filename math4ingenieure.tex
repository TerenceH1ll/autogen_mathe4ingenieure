\documentclass[a4paper,12pt]{article}
\usepackage{fancyhdr}
%\usepackage{fancyheadings}
\usepackage[ngerman,german]{babel}
\usepackage{german}
\usepackage[utf8]{inputenc}
%\usepackage[latin1]{inputenc}
\usepackage[active]{srcltx}
\usepackage{algorithm}
\usepackage[noend]{algorithmic}
\usepackage{amsmath}
\usepackage{amssymb}
\usepackage{amsthm}
\usepackage{bbm}
\usepackage{enumerate}
\usepackage{graphicx}
\usepackage{ifthen}
\usepackage{listings}
\usepackage{struktex}
\usepackage{hyperref}
\usepackage[shortlabels]{enumitem}
\usepackage{pythontex}
\usepackage{mathtools}

\usepackage{enumitem,multicol,setspace}% http://ctan.org/pkg/{enumitem,multicol,setspace}
\newcounter{subenum}[enumi]
\renewcommand{\thesubenum}{\alph{subenum}}
\newcommand{\newitem}[1]{%
  \refstepcounter{subenum}%
  \parbox{\dimexpr.5\linewidth-.5\columnsep}{%
    \makebox[\labelwidth][r]{(\thesubenum)\hspace*{\labelsep}}%
    #1}\hfill%
}

%%%%%%%%%%%%%%%
%% Aufgaben-COMMAND
\newcommand{\Aufgabe}[1]{
	{
		\vspace*{0.5cm}
		\textsf{\textbf{Aufgabe #1}}
		\vspace*{0.2cm}
		
	}
}
%%%%%%%%%%%%%%

%%%%%%%%%%%%%%%%%%%%%%
%% Lösungen Einblenden COMMAND
\newcount\MitLoesungen % Hilfsvariable um  einzustellen ob Lösungen mitgedruckt werden
\MitLoesungen=1 % 1 - Lösungen werden mitgedruckt 0 - nicht
%%%%%%%%%%%%%%%%%%%%%%

\pagenumbering{gobble}

%%%%%%%%%%%%%%%%%%%%%%%%%%%%%%%%%%%%%%%%%%%%%%%%%%%%%%
%%%%%%%%%%%%%% EDIT THIS PART %%%%%%%%%%%%%%%%%%%%%%%%
%%%%%%%%%%%%%%%%%%%%%%%%%%%%%%%%%%%%%%%%%%%%%%%%%%%%%%
\newcommand{\Fach}{Mathematik I für Bachelor Maschinenbau}
\newcommand{\Name}{Felix Frank, Tobias Czernek}
\newcommand{\Hochschule}{Hochschule Darmstadt}
%\newcommand{\Matrikelnummer}{1}
\newcommand{\Semester}{Sommersemester 2019}
%%%%%%%%%%%%%%%%%%%%%%%%%%%%%%%%%%%%%%%%%%%%%%%%%%%%%%
%%%%%%%%%%%%%%%%%%%%%%%%%%%%%%%%%%%%%%%%%%%%%%%%%%%%%%

%%%%%%%%%%%%%%%%%%%%%%%%%%%%%%%%%
%%Elementare Umformungen
%%%%%%%%%%%%%%%%%%%%%%%%%%%%%%%%%

\begin{document}
	\thispagestyle{fancy}
	\lhead{\sf \large \Fach{} \vspace{0,1cm}\\ \small \Name{}}% - \Matrikelnummer{}}
	\rhead{\sf \Hochschule{} \\ \Semester{}}
	\vspace*{0.2cm}
	\begin{center}
		\LARGE \sf \textbf{Tutoriumsaufgaben }
	\end{center}
	\vspace*{0.2cm}

%%%%%%%%%%%%%%%%%%%%%%%%%%%%%%%%%
%%	Version hochsetzen um Pythonwerte neu zu kompilieren
%%%%%%%%%%%%%%%%%%%%%%%%%%%%%%%%%
\begin{pycode}
version = 5
\end{pycode}

%%%%%%%%%%%%%%%%%%%%%%%%%%%%%%%%%
%%	AUFGABE 1
%%%%%%%%%%%%%%%%%%%%%%%%%%%%%%%%%
\begin{pycode}
from random import choice, randint
a = choice([2,3,5])
b = randint(1,3)
c = randint(1,3)
d = randint(2,3)
e = randint(2,5)
f = randint(2,5)
g = randint(3,5)
h = randint(3,5)
i = randint(3,5)
j = randint(4,5)
\end{pycode}

\Aufgabe{1}
Vereinfachen Sie die bitte die folgenden Terme
\setcounter{subenum}{0}
\vspace{0,2cm}\\
\setstretch{2}
\newitem{$\begin{aligned}[t]
	\frac{(\py{a**b}^{\py{e}})^{\py{h}} \cdot (\py{a**c}^{\py{f}})^{\py{i}}}{(\py{a**d}^{\py{g}})^{\py{j}}}
	%\frac{(12^2)^4 \cdot (8^4)^3}{(4^4)^6}
	\end{aligned}$ 
	\newline
	\ifnum\MitLoesungen=1 
	Richtige Antwort: \boxed{\py{a**b}}.
	\fi 
	\newline }
\newitem{$\begin{aligned}[t]
	(\frac{4a^{-2}x}{3a^5x^{-3}})^2 : \frac{(3a^4x^2)^{-3}}{(2ax^{-3})^{-2}}
	\end{aligned}$ }
\newitem{$\begin{aligned}[t]
	\frac{3-a}{a^{m-4}} + \frac{a^6 - a^5 + 2a^3 - 1}{a^{m+1}} - \frac{2a^2+1}{a^{m-2}}
	\end{aligned}$ }
\setstretch{1}\\



%%%%%%%%%%%%%%%%%%%%%%%%%%%%%%%%%
%%	AUFGABE 2
%%%%%%%%%%%%%%%%%%%%%%%%%%%%%%%%%
\begin{pycode}
from random import choice, randint
import math
a = choice([2,3,4,5])
b = randint(0,5)
c = randint(0,5)
x = math.log((a**b)/(a**c), a)
\end{pycode}

\Aufgabe{2}
Bestimmen Sie bitte $x$ ohne Hilfsmittel:
\setcounter{subenum}{0}
\vspace{0,2cm}\\
\setstretch{2}
\newitem{$\begin{aligned}[t]
	x = \log_{\py{a}}{\frac{\py{a**b}}{\py{a**c}}}
	%x = \log_2{\frac{1}{8}}
	\end{aligned}$ 
	\newline
	\ifnum\MitLoesungen=1 
	Richtige Antwort: \boxed{\pyc{print(int(x))}}.
	\fi 
	\newline }
\newitem{$\begin{aligned}[t]
	\log_x{\sqrt{8}} = \frac{3}{4}
	\end{aligned}$ }
\newitem{$\begin{aligned}[t]
	x = 81^{0,5 \cdot \log_3{7}}
	\end{aligned}$ } 
\newitem{$\begin{aligned}[t]
	x = 2\cdot10^{2\cdot \log_{10}{2}}
	\end{aligned}$ }
\newitem{$\begin{aligned}[t]
	x = \sqrt[3]{10^{\frac{1}{2}(\log_{10}{2}+\log_{10}{32})}}
	\end{aligned}$ }
\newitem{$\begin{aligned}[t]
	x = \sqrt{10^{\log_{10}{16}}}
	\end{aligned}$ }
\setstretch{1}\\

%%%%%%%%%%%%%%%%%%%%%%%%%%%%%%%%%
%%	AUFGABE 3
%%%%%%%%%%%%%%%%%%%%%%%%%%%%%%%%%
\begin{pycode}
from random import choice, randint
import math
from sympy.solvers.inequalities import reduce_inequalities
from sympy import Symbol, sin, Interval, S
x = Symbol('x')
a = choice([2,3,4,5])
b = randint(3,5)
c = randint(1,2)
y = reduce_inequalities((a*x-b)/a > (a*x-c)/b, x)
\end{pycode}

\Aufgabe{3}
Bestimmen Sie bitte die Lösungsmengen der folgenden Ungleichungen
\setcounter{subenum}{0}
\vspace{0,2cm}\\
\setstretch{2}
\newitem{$\begin{aligned}[t]
	\frac{\py{a}x-\py{b}}{\py{a}} < \frac{\py{a}x-\py{c}}{-\py{b}}
	%\frac{2x-3}{2} < \frac{2x-1}{-3}
	\end{aligned}$ 
	\newline
	\ifnum\MitLoesungen=1 
	Richtige Antwort: \boxed{\py{y}}.
	\fi 
	\newline }
\newitem{$\begin{aligned}[t]
	\frac{2x-1}{2} \leq x+1
	\end{aligned}$ }
\newitem{$\begin{aligned}[t]
	\left| 2x-3 \right| \leq 6
	\end{aligned}$ }
\newitem{$\begin{aligned}[t]
	-x^2 + \left| 2x+4 \right| \geq 1
	\end{aligned}$ }
\setstretch{1}\\

%%%%%%%%%%%%%%%%%%%%%%%%%%%%%%%%%
%Vektoren
%%%%%%%%%%%%%%%%%%%%%%%%%%%%%%%%%

%%%%%%%%%%%%%%%%%%%%%%%%%%%%%%%%%
%%	AUFGABE 4
%%%%%%%%%%%%%%%%%%%%%%%%%%%%%%%%%
\begin{pycode}
from random import choice, randint
import math
from numpy import array
a = randint(1,3)
b = randint(1,3)
c = randint(4,6)
d = randint(4,6)
x = array([c,a])
y = array([b,d])
z = 2*x-y
\end{pycode}

\Aufgabe{4}
%Gegeben seien die Vektoren $\vec{a} = (2,1)^T$ und $\vec{b} = (1,3)^T$. Berechnen Sie bitte $2\vec{a}-\vec{b}$ rechnerisch und zeichnerisch.
Gegeben seien die Vektoren $\vec{a} = (\py{c},\py{a})^T$ und $\vec{b} = (\py{b},\py{d})^T$. Berechnen Sie bitte $2\vec{a}-\vec{b}$ rechnerisch und zeichnerisch.
\newline
\ifnum\MitLoesungen=1 
Richtige Antwort: \boxed{\begin{pmatrix} \py{z[0]} \\ \py{z[1]} \end{pmatrix}}.
\fi 
\newline

%%%%%%%%%%%%%%%%%%%%%%%%%%%%%%%%%
%%	AUFGABE 5
%%%%%%%%%%%%%%%%%%%%%%%%%%%%%%%%%
\begin{pycode}
from random import choice, randint
import math
from numpy import array
a = randint(1,3)
b = randint(1,3)
c = randint(4,6)
d = randint(4,6)
y = array([b,d])
\end{pycode}

\Aufgabe{5}
Untersuchen Sie bitte die folgenden Vektoren auf lineare Abhängigkeit
\begin{enumerate}[(a)]
	\item $\begin{aligned}[t]
	\vec{a} = (\py{a},\py{a},\py{a})^T,\vec{b} = (\py{a+1},\py{b+1},\py{a+1})^T,\vec{c} = (\py{c},\py{d},\py{c})^T
	%\vec{a} = (1,1,1)^T,\vec{b} = (1,2,1)^T,\vec{c} = (3,5,3)^T	
	\end{aligned}$
	\newline
	\ifnum\MitLoesungen=1 
	Richtige Antwort: \boxed{???}.
	\fi 
	\newline
	\item $\begin{aligned}[t]
	\vec{a} = (1,0,1)^T,\vec{b} = (1,2,0)^T,\vec{c} = (0,5,3)^T
	\end{aligned}$
\end{enumerate}

\Aufgabe{6}
\begin{enumerate}
	\item Bestimmen Sie bitte alle Vektoren vom Betrag 3, die auf dem Vektor $\vec{a} = (-4,3)^T$ senkrecht stehen.
	\item Der Vektor $\vec{a} = (-3,2,-4)^T$ wird auf den Vektor $\vec{b} = (1,1,1)^T$ orthogonal projiziert. Welche Länge hat dann seine Projektion? 
	\item Gegeben seien die Vektoren $\vec{a} = (2,y,z)^T, \vec{b} = (-1,4,2)^T$ und $\vec{c} = (3,-3,-1)^T$. Bestimmen Sie bitte $y$ und $z$ so, dass der Vektor $\vec{a}$ auf $\vec{b}$ und $\vec{c}$ senkrecht steht. Welchen Betrag hat $\vec{a}$, und welchen Winkel bildet er mit den Vektoren $\vec{b}+\vec{c}$ und $\vec{a}+\vec{b}+\vec{c}$?
\end{enumerate}

%%%%%%%%%%%%%%%%%%%%%%%%%%%%%%%%%
%Matritzen und LGS
%%%%%%%%%%%%%%%%%%%%%%%%%%%%%%%%%
%%%%%%%%%%%%%%%%%%%%%%%%%%%%%%%%%
%%	AUFGABE 7
%%%%%%%%%%%%%%%%%%%%%%%%%%%%%%%%%
\begin{pycode}
from random import choice, randint
import math
from numpy import array
a = randint(1,3)
b = randint(1,3)
c = randint(1,3)
d = randint(1,3)
e = randint(4,6)
f = randint(4,6)
g = randint(4,6)
h = randint(4,6)
x = array([[ e, a ,b], [ f, c ,d]])
y = array([[ a, g ,b], [ c, d ,h], [ e, a ,a]])
z = x.dot(y)
\end{pycode}

\Aufgabe{7}
Berechnen Sie bitte das Matrizenprodukt $C = AB$ der folgenden Matrizen:
\setcounter{subenum}{0}
\vspace{0,2cm}\\
\setstretch{2}
\newitem{$\begin{aligned}[t]
	A = \begin{pmatrix} \py{e} &\py{a} &\py{b}\\\py{f} &\py{c} &\py{d} \end{pmatrix}, B = \begin{pmatrix} \py{a} &\py{g} &\py{b}\\\py{c} &\py{d} &\py{h}\\\py{e} &\py{a} & \py{a} \end{pmatrix}
	%A = \begin{pmatrix} 5 &2 &1\\3 &0 &1 \end{pmatrix}, B = \begin{pmatrix} 1 &3 &0\\1 &1 &4\\2 &0 & 0 \end{pmatrix}
	\end{aligned}$ 
	\newline
	\ifnum\MitLoesungen=1 
	Richtige Antwort: \boxed{\begin{pmatrix} \py{z[0][0]} & \py{z[0][1]} & \py{z[0][2]} \\ \py{z[1][0]} & \py{z[1][1]} & \py{z[1][2]} \end{pmatrix}}
	\fi 
	\newline } \\
\newitem{$\begin{aligned}[t]
	A = \begin{pmatrix} 5 &2 &1 \end{pmatrix}, B = \begin{pmatrix} 1\\3\\2 \end{pmatrix}
	\end{aligned}$ } \\
\setstretch{1}\\ 


%%%%%%%%%%%%%%%%%%%%%%%%%%%%%%%%%
%%	AUFGABE 8
%%%%%%%%%%%%%%%%%%%%%%%%%%%%%%%%%
\begin{pycode}
from random import choice, randint
from sympy import *
import math
import numpy as np
a = randint(1,3)
b = randint(1,3)
c = randint(4,6)
d = randint(4,6)
x = np.array([[ a, c],[ d, b]])
y = np.array([[ b, -c],[ -d, a]])
detx = np.linalg.det(x)
if detx != 0:
	test = "regulary"
else:
	test = "singulary"
\end{pycode}

\Aufgabe{8}
Bestimmen Sie bitte die Inversen der folgenden Matrizen:
\setcounter{subenum}{0}
\vspace{0,2cm}\\
\setstretch{2}
\newitem{$\begin{aligned}[t]
	A = \begin{pmatrix} \py{a} & \py{c} \\ \py{d} & \py{b} \end{pmatrix}
	%A = \begin{pmatrix} 1 &3\\2 &1 \end{pmatrix}
	\end{aligned}$ 
	\newline
	\ifnum\MitLoesungen=1 
	\ifthenelse{\equal{singulary}{\py{test}}}{Richtige Antwort: \boxed{Die Matrix ist singulär}}{Richtige Antwort: \boxed{\begin{pmatrix} \py{y[0][0]} & \py{y[0][1]} \\ \py{y[1][0]} & \py{y[1][1]} \end{pmatrix}}}
	\fi 
	\newline }
\setstretch{1}\\ 

%%%%%%%%%%%%%%%%%%%%%%%%%%%%%%%%%
%%	AUFGABE 9
%%%%%%%%%%%%%%%%%%%%%%%%%%%%%%%%%
\begin{pycode}
from random import choice, randint
a = randint(1,3)
b = choice([2,3,4,5])
\end{pycode}

\Aufgabe{9}
Beurteilen Sie bitte anhand des freien Parameters $a \in \mathbb{R}$ die L\"osbarkeit des linearen Gleichungssystems $A\vec{x} = \vec{b}$.
\setcounter{subenum}{0}
\vspace{0,2cm}\\
\setstretch{2}
\newitem{$\begin{aligned}[t]
	A = \begin{pmatrix} a-\py{b} &\py{a} &\py{a}\\\py{a} &a-\py{b} &\py{a}\\\py{a} &\py{a} &a-\py{b} \end{pmatrix}, \vec{b} = \begin{pmatrix} \py{a}\\\py{a}\\\py{a} \end{pmatrix}
	%A = \begin{pmatrix} a-2 &1 &1\\1 &a-2 &1\\1 &1 &a-2 \end{pmatrix}, \vec{b} = \begin{pmatrix} 1\\1\\1 \end{pmatrix}
	\end{aligned}$ 
	\newline
	\ifnum\MitLoesungen=1 
	Richtige Antwort: \boxed{???}.
	\fi 
	\newline } \\
\setstretch{1}\\ 

%%%%%%%%%%%%%%%%%%%%%%%%%%%%%%%%%
%Funktionen
%%%%%%%%%%%%%%%%%%%%%%%%%%%%%%%%%
%%%%%%%%%%%%%%%%%%%%%%%%%%%%%%%%%
%%	AUFGABE 10
%%%%%%%%%%%%%%%%%%%%%%%%%%%%%%%%%
\begin{pycode}
from random import choice, randint
a = randint(2,5)
\end{pycode}

\Aufgabe{10}
Bestimmen Sie bitte das Symmetrieverhalten und den maximalen Definitionsbereich $D(f)$ der folgenden Funktionen
\setcounter{subenum}{0}
\vspace{0,2cm}\\
\setstretch{2}
\newitem{$\begin{aligned}[t]
	f(x) = \py{a} x^2 - \py{a*a}
	%f(x) = 4x^2- 16
	\end{aligned}$ 
	\newline
	\ifnum\MitLoesungen=1 
	Richtige Antwort: \boxed{???}.
	\fi 
	\newline }
\newitem{$\begin{aligned}[t]
	f(x) = \frac{x^3}{x^2+1}
	\end{aligned}$ }
\newitem{$\begin{aligned}[t]
	f(x) = \sin(x)\cos(x)
	\end{aligned}$ }
\newitem{$\begin{aligned}[t]
	f(x) = \sqrt{x^2 - 25}
	\end{aligned}$ }
\setstretch{1}\\ 


%%%%%%%%%%%%%%%%%%%%%%%%%%%%%%%%%
%%	AUFGABE 11
%%%%%%%%%%%%%%%%%%%%%%%%%%%%%%%%%
\begin{pycode}
from random import choice, randint
a = randint(1,5)
\end{pycode}

\Aufgabe{11}
Berechnen Sie von den folgenden Funktionen die Umkehrfunktion $f^{-1}$
\setcounter{subenum}{0}
\vspace{0,2cm}\\
\setstretch{2}
\newitem{$\begin{aligned}[t]
	f(x) = \frac{\py{a}}{\py{2*a}x} \text{ mit } x > 0
	%f(x) = \frac{1}{2x} \text{ mit } x > 0
	\end{aligned}$ 
	\newline
	\ifnum\MitLoesungen=1 
	Richtige Antwort: \boxed{???}.
	\fi 
	\newline }
\newitem{$\begin{aligned}[t]
	f(x) = \sqrt{3x} \text{ mit } x > 0
	\end{aligned}$ }
\newitem{$\begin{aligned}[t]
	f(x) = 2e^{x-0,5}
	\end{aligned}$ }
\setstretch{1}\\ 

%%%%%%%%%%%%%%%%%%%%%%%%%%%%%%%%%
%Ableitung
%%%%%%%%%%%%%%%%%%%%%%%%%%%%%%%%%
%%%%%%%%%%%%%%%%%%%%%%%%%%%%%%%%%
%%	AUFGABE 12
%%%%%%%%%%%%%%%%%%%%%%%%%%%%%%%%%
\begin{pycode}
from random import choice
c = choice(['\sin','\cos','\ttan'])
d = choice(['+1','-1',''])
\end{pycode}

\Aufgabe{12}
Berechnen Sie bitte den Grenzwert der folgenden Funktionen
\setcounter{subenum}{0}
\vspace{0,2cm}\\
\setstretch{2}
\newitem{$\begin{aligned}[t]
	\lim_{x\rightarrow \pm \infty }{\frac {x\cdot {\py{c}{\left({x}\right)}}}{{x}^{{2}}\py{d}}}
	%\lim_{x\rightarrow \pm \infty }{\frac {x\cdot {\sin{\left({x}\right)}}}{{x}^{{2}}+1}}
	\end{aligned}$ 
	\newline
	\ifnum\MitLoesungen=1 
	Richtige Antwort: \boxed{???}.
	\fi 
	\newline }
\newitem{$\begin{aligned}[t]
	\lim_{x\rightarrow \infty }{\frac {x+1}{x-1}}\cdot \frac {{x}^{{n}}-1}{{x}^{{n}}+1}
	\end{aligned}$ }
\newitem{$\begin{aligned}[t]
	\lim_{x\rightarrow \infty }{\sqrt{{x+1}}}-\sqrt{{x}}
	\end{aligned}$ }
\newitem{$\begin{aligned}[t]
	\lim_{x\rightarrow -1}{\frac {\left({x+3}\right)\left({2x-1}\right)}{{x}^{{2}}+3x-2}}
	\end{aligned}$ }
\newitem{$\begin{aligned}[t]
	\lim_{h\rightarrow 0}{\frac {{{\left({x+h}\right)}}^{{3}}-{x}^{{3}}}{h}}
	\end{aligned}$ }
\setstretch{1}\\ 


%%%%%%%%%%%%%%%%%%%%%%%%%%%%%%%%%
%%	AUFGABE 13
%%%%%%%%%%%%%%%%%%%%%%%%%%%%%%%%%
\begin{pycode}
from random import choice, randint
a = randint(1,5)
b = randint(1,5)
c = randint(1,5)
d = randint(1,5)
e = randint(1,5)
f = randint(1,5)
g = randint(1,5)
h = choice(['+','-'])
i = choice(['+','-'])
j = choice(['+','-'])
\end{pycode}

\Aufgabe{13}
Berechnen Sie bitte die Ableitung der Funktion $f(x)$
\setcounter{subenum}{0}
\vspace{0,2cm}\\
\setstretch{2}
\newitem{$\begin{aligned}[t]
	f\left({x}\right)=\py{a}{x}^{\py{b}} \py{h} \py{c}{x}^{\py{d}} \py{i} \py{e}{x}^{\py{f}} \py{j} \py{g}
	%f\left({x}\right)=3{x}^{{4}}-{x}^{{3}}+2{x}^{{2}}-5
	\end{aligned}$
	\newline
	\ifnum\MitLoesungen=1 
	Richtige Antwort: \boxed{???}.
	\fi 
	\newline }
\newitem{$\begin{aligned}[t]
	f\left({x}\right)=2+\frac {3}{{x}^{{4}}}
	\end{aligned}$ }
\newitem{$\begin{aligned}[t]
	f\left({x}\right)=\frac {x+8}{x-8}
	\end{aligned}$ }
\newitem{$\begin{aligned}[t]
	f\left({x}\right)={\ln{\left({\frac {{x}^{{4}}}{{{\left({3x-4}\right)}}^{{2}}}}\right)}}
	\end{aligned}$ }
\newitem{$\begin{aligned}[t]
	f\left({x}\right)=\sqrt{{{x}^{{2}}+6x+3}}
	\end{aligned}$ }
\setstretch{1}\\ 


%%%%%%%%%%%%%%%%%%%%%%%%%%%%%%%%%
%%	AUFGABE 14
%%%%%%%%%%%%%%%%%%%%%%%%%%%%%%%%%
\begin{pycode}
from random import choice, randint
a = randint(2,5)
b = randint(2,5)
c = randint(5,10)
d = randint(5,10)
e = choice(['+','-'])
f = choice(['+','-'])
g = choice(['+','-'])
\end{pycode}

\Aufgabe{14}
Bestimmen Sie von der Funktion $f(x)$ bitte die Extremstellen
\setcounter{subenum}{0}
\vspace{0,2cm}\\
\setstretch{2}
\newitem{$\begin{aligned}[t]
	f(x) = \frac {1}{\py{a}}{x}^{{\py{a}}} \py{e} \frac {1}{\py{b}}{x}^{{\py{b}}} \py{f} \py{c}x \py{g} \py{d}
	%f(x) = \frac {1}{3}{x}^{{3}}+\frac {1}{2}{x}^{{2}}-6x+8
	\end{aligned}$
	\newline
	\ifnum\MitLoesungen=1 
	Richtige Antwort: \boxed{???}.
	\fi 
	\newline }
\newitem{$\begin{aligned}[t]
	f(x) = \frac {10}{x^2+1}
	\end{aligned}$ }
\setstretch{1}\\


%%%%%%%%%%%%%%%%%%%%%%%%%%%%%%%%%
%%	AUFGABE 15
%%%%%%%%%%%%%%%%%%%%%%%%%%%%%%%%%
\begin{pycode}
from random import choice, randint
a = randint(2,4)
b = randint(3,8)
c = randint(3,8)
d = choice(['+','-'])
e = choice(['+','-'])
\end{pycode}

\Aufgabe{15}
Bestimmen Sie bitte von der Funktion $f(x)$ die Wendepunkte
\setcounter{subenum}{0}
\vspace{0,2cm}\\
\setstretch{2}
\newitem{$\begin{aligned}[t]
	f(x) = \py{c}x \py{e} (x \py{d} \py{b})^{\frac{1}{\py{a}}}
	%f(x) = 5x - (x-4)^{\frac{1}{3}}
	\end{aligned}$ 
	\newline
	\ifnum\MitLoesungen=1 
	Richtige Antwort: \boxed{???}.
	\fi 
	\newline }
\newitem{$\begin{aligned}[t]
	f(x) = (1-2x)^3
	\end{aligned}$ }
\setstretch{1}\\ 

%%%%%%%%%%%%%%%%%%%%%%%%%%%%%%%%%
%%	AUFGABE 16
%%%%%%%%%%%%%%%%%%%%%%%%%%%%%%%%%
\begin{pycode}
from random import choice, randint
a = randint(2,5)
b = randint(2,5)
c = randint(3,8)
d = choice(['+','-'])
e = choice(['+','-'])
\end{pycode}

\Aufgabe{16}
Berechnen Sie bitte die Grenzwerte folgender Funktionen mit Hilfe der Regel von L'Hospital
\setcounter{subenum}{0}
\vspace{0,2cm}\\
\setstretch{2}
\newitem{$\begin{aligned}[t]
	\lim_{x\rightarrow 1}{\frac {-\frac {1}{4}+x+\frac {1}{2}{x}^{\py{a}}\left({-\frac {3}{2}+{\ln{\left({x}\right)}}}\right)}{{{\left({x-1}\right)}}^{\py{a}}}}
	%\lim_{x\rightarrow 1}{\frac {-\frac {1}{4}+x+\frac {1}{2}{x}^{{2}}\left({-\frac {3}{2}+{\ln{\left({x}\right)}}}\right)}{{{\left({x-1}\right)}}^{{3}}}}
	\end{aligned}$
	\newline
	\ifnum\MitLoesungen=1 
	Richtige Antwort: \boxed{???}.
	\fi 
	\newline }
\newitem{$\begin{aligned}[t]
	\lim_{x\rightarrow 0}{\frac {{e }^{{3x}}-1}{x}}
	\end{aligned}$ }
\newitem{$\begin{aligned}[t]
	\lim_{x\rightarrow 1}{\frac {1-{\sin{\left({\frac {\pi }{2}x}\right)}}}{3{x}^{{2}}-6x+3}}
	\end{aligned}$ }
\newitem{$\begin{aligned}[t]
	\lim_{x\rightarrow 0}{\frac {{e }^{{x}}+{e }^{{-x}}-2}{x-{\ln{\left({1+x}\right)}}}}
	\end{aligned}$ }
\setstretch{1}\\ 

%%%%%%%%%%%%%%%%%%%%%%%%%%%%%%%%%
%Integral
%%%%%%%%%%%%%%%%%%%%%%%%%%%%%%%%%
%%%%%%%%%%%%%%%%%%%%%%%%%%%%%%%%%
%%	AUFGABE 17
%%%%%%%%%%%%%%%%%%%%%%%%%%%%%%%%%
\begin{pycode}
from random import choice, randint
a = randint(2,5)
b = randint(2,5)
\end{pycode}

\Aufgabe{17}
Lösen Sie folgende \textit{unbestimmte} Integrale:
\begin{multicols}{2}
	\begin{enumerate}[a)]
		\item $\int  \sqrt[\py{a}]{x^\py{b}}dx$
		%\item $\int  \sqrt[3]{x^2}dx$
		\ifnum\MitLoesungen=1 
		\newline Richtige Antwort: \boxed{???}.
		\fi 
		\item $\int  (2x^4+3x^2-4x+7)dx$
		\item $\int  \sqrt{x \sqrt{x}}$
		\item $\int  (cos(2x+5)- e^{-2x})dx$
		
		
	\end{enumerate}
\end{multicols}
\textbf{Tipps:}\\
a) - c) Potenzfunktionen, Summenregel\\
d) Summenregel, inv. Kettenregel\\

%%%%%%%%%%%%%%%%%%%%%%%%%%%%%%%%%
%%	AUFGABE 18
%%%%%%%%%%%%%%%%%%%%%%%%%%%%%%%%%
\begin{pycode}
from random import choice, randint
a = randint(2,5)
\end{pycode}

\Aufgabe{18}
Berechnen Sie bitte die folgenden Integrale mit \textit{partieller Integration}
\begin{multicols}{2}
	\begin{enumerate}[a)]
		\item $\int  x^{\py{a}} ln(x)dx$
		%\item $\int  x^3 ln(x)dx$
		\ifnum\MitLoesungen=1 
		\newline Richtige Antwort: \boxed{???}.
		\fi 
		\item $\int  x^2 e^{2x}dx$
		\item $\int  \frac{x}{\sqrt{x+1}} dx$
		\item $\int  x^2 sin(ln(x))dx$
		
	\end{enumerate}
\end{multicols}
\textbf{Tipps:}\\
b) 2x partiell integrieren; d) 2x partiell integrieren und Gleichung umstellen

%%%%%%%%%%%%%%%%%%%%%%%%%%%%%%%%%
%%	AUFGABE 19
%%%%%%%%%%%%%%%%%%%%%%%%%%%%%%%%%
\begin{pycode}
from random import choice, randint
a = randint(5,9)
b = randint(5,9)
c = randint(2,5)
d = randint(2,5)
\end{pycode}

\Aufgabe{19}
Berechnen Sie bitte die folgenden Integrale mit geeigneter \textit{Substitution}
\begin{multicols}{2}
	\begin{enumerate}[a)]
		\item $\int  \py{a**2}x (\py{b}x^{\py{c}}-\py{d})^{\py{d}}  dx$
		%\item $\int  56x (7x^2-3)^3  dx$
		\ifnum\MitLoesungen=1 
		\newline Richtige Antwort: \boxed{???}.
		\fi 
		\item $\int  \frac{8x^2}{(x^3+2)^3} dx$
		\item $\int  x\sqrt{1-2x^2} dx$
		\item $\int  \frac{x}{x^2+1} dx$
		\item $\int  \frac{x^3}{x^4+1} dx$
		\item $\int x^2  e^{x^3-2} dx$
	\end{enumerate}
\end{multicols}

%%%%%%%%%%%%%%%%%%%%%%%%%%%%%%%%%
%%	AUFGABE 20
%%%%%%%%%%%%%%%%%%%%%%%%%%%%%%%%%
\begin{pycode}
from random import choice, randint
a = randint(3,5)
b = randint(1,3)
c = randint(1,5)
d = randint(1,5)
e = randint(1,5)
f = randint(1,5)
g = choice(['+','-'])
h = choice(['+','-'])
\end{pycode}

\Aufgabe{20}
Berechnen Sie bitte die folgenden \textit{bestimmten} Integrale
\begin{multicols}{2}
	\begin{enumerate}[a)]
		\item $\int^{\py{a}}_{\py{b}}  \frac{\py{c}x^{\py{f}} \py{g} \py{d}x^{\py{f-1}} \py{h} \py{e}}{x^{\py{f-1}}}  dx$
		%\item $\int^{2}_{1}  \frac{x^3+5x^2-4}{x^2}  dx$
		\ifnum\MitLoesungen=1 
		\newline Richtige Antwort: \boxed{???}.
		\fi
		\item $\int^{1}_{0}  (1-x)\sqrt{x} dx$
		\item $\int^{2\pi}_{0} (sin(\frac{\pi}{2})-e^{3x}) dx$
	\end{enumerate}
\end{multicols}

%%%%%%%%%%%%%%%%%%%%%%%%%%%%%%%%%
%Funktionn Mehreren Variabeln: Ableitung/Integral
%%%%%%%%%%%%%%%%%%%%%%%%%%%%%%%%%
%%%%%%%%%%%%%%%%%%%%%%%%%%%%%%%%%
%%	AUFGABE 21
%%%%%%%%%%%%%%%%%%%%%%%%%%%%%%%%%
\begin{pycode}
from random import choice, randint
a = randint(1,4)
\end{pycode}

\Aufgabe{21}
Berechnen Sie bitte die folgenden \textit{Mehrfachintegrale}:
\begin{multicols}{2}
	\begin{enumerate}[a)]
		\item $\int^{\py{a}}_{\py{a-1}} \int^{\py{a}-x}_{\py{a-1}} (x+y) dy dx$
		%\item $\int^{1}_{0} \int^{1-x}_{0} (x+y) dy dx$
		\ifnum\MitLoesungen=1 
		\newline Richtige Antwort: \boxed{???}.
		\fi
		\item $\int^{1}_{0} \int^{x^2}_{0} 2x e^y dy dx$
		\item $\int^{2\pi}_{0} \int^{1}_{0} x e^{x^2}dx dy$
		\item $\int^{2}_{1} \int^{2}_{1} \sqrt{1+x+y} dy dx$	
	\end{enumerate}
\end{multicols}

%%%%%%%%%%%%%%%%%%%%%%%%%%%%%%%%%
%%	AUFGABE 22
%%%%%%%%%%%%%%%%%%%%%%%%%%%%%%%%%
\begin{pycode}
from random import choice, randint
a = randint(1,5)
\end{pycode}

\Aufgabe{22}
\begin{enumerate}[a)]
	
	\item Ein Rechteck in der x-y-Ebene ist durch  die Ungleichungen $0\leq x \leq \py{a}$ und $0 \leq y \leq \py{a/2}$ begrenzt. Berechne das Volumen des Körpers, der über dem Rechteck und unter der Funktion $z= \py{a/2} +xy$ liegt. 
	%\item Ein Rechteck in der x-y-Ebene ist durch  die Ungleichungen $0\leq x \leq 2$ und $0 \leq y \leq 1$ begrenzt. Berechne das Volumen des Körpers, der über dem Rechteck und unter der Funktion $z= 1 +xy$ liegt. %Lösung=3
	\ifnum\MitLoesungen=1 
		\newline Richtige Antwort: \boxed{???}.
	\fi
	\item Berechnen Sie bitte die Fläche, die von der Geraden $y=x$ und der Parabel $y=x^2-2x$ eingeschlossen wird. %Lösung = 9/2
	\item Berechnen Sie bitte die von den Kurven $y^2=2-x$ und $y=x$ eingeschlossene Fläche mit Hilfe eines Doppelintegrals.
	\item Berechnen Sie bitte das Volumen des Körpers, der von den Koordinatenebenen
	und der Ebene $x+y+z=1$ begrenzt wird mit Hilfe eines Doppelintegrals.
	
\end{enumerate}

%%%%%%%%%%%%%%%%%%%%%%%%%%%%%%%%%
%%	AUFGABE 23
%%%%%%%%%%%%%%%%%%%%%%%%%%%%%%%%%
\begin{pycode}
from random import choice, randint
a = randint(1,5)
b = randint(1,5)
c = randint(1,5)
d = randint(1,5)
e = randint(1,5)
\end{pycode}

\Aufgabe{23}
\begin{enumerate}[a)]
	\item Wo liegt der Schwerpunkt der von den Funktionen $f (x) = \py{a}x^2 + \py{b}x + \py{c}$ und $g(x) = \py{d}x + \py{e}$ eingeschlossenen Fläche?
	%\item Wo liegt der Schwerpunkt der von den Funktionen $f (x) = x^2 + 2x + 1$ und $g(x) = 3x + 1$ eingeschlossenen Fläche?
	\ifnum\MitLoesungen=1 
		\newline Richtige Antwort: \boxed{???}.
	\fi
	\item Berechnen Sie bitte den Schwerpunkt der von den Kurven $f(x)=ln(x)$,$g(x)=0.1x-0.1$ und $x = 5$ begrenzten Fläche.
\end{enumerate}

%%%%%%%%%%%%%%%%%%%%%%%%%%%%%%%%%
%Komplexe Zahlen
%%%%%%%%%%%%%%%%%%%%%%%%%%%%%%%%%
%%%%%%%%%%%%%%%%%%%%%%%%%%%%%%%%%
%%	AUFGABE 24
%%%%%%%%%%%%%%%%%%%%%%%%%%%%%%%%%
\begin{pycode}
from random import choice, randint
a = randint(2,5)
b = randint(2,5)
\end{pycode}

\Aufgabe{24}
Geben Sie die bitte die trigonometrische und kartesische Form der folgenden komplexen Zahlen an
\setcounter{subenum}{0}
\vspace{0,2cm}\\
\setstretch{2}
\newitem{$\begin{aligned}[t]
	\py{a}\sqrt{2}e^{i\frac{\pi}{\py{b}}}
	%3\sqrt{2}e^{i\frac{\pi}{4}}
	\end{aligned}$ 
	\newline
	\ifnum\MitLoesungen=1 
	Richtige Antwort: \boxed{???}.
	\fi 
	\newline }
\newitem{$\begin{aligned}[t]
	2e^{i\frac{2\pi}{3}}
	\end{aligned}$ }
\newitem{$\begin{aligned}[t]
	e^{i\pi}
	\end{aligned}$ }
\newitem{$\begin{aligned}[t]
	4e^{i\frac{4\pi}{3}}
	\end{aligned}$ }
\newitem{$\begin{aligned}[t]
	5e^{i\frac{\pi}{2}}
	\end{aligned}$ }
\setstretch{1}\\  

%%%%%%%%%%%%%%%%%%%%%%%%%%%%%%%%%
%%	AUFGABE 25
%%%%%%%%%%%%%%%%%%%%%%%%%%%%%%%%%
\begin{pycode}
from random import choice, randint
a = randint(3,5)
b = randint(1,3)
c = randint(3,5)
d = randint(1,3)
e = randint(1,5)
f = choice(['','-'])
g = choice(['+','-'])
h = choice(['','-'])
i = choice(['+','-'])
\end{pycode}

\Aufgabe{25}
Es seien $z_1 = \py{f}\py{a}i, z_2 = \py{c}\py{g}\py{b}i$ und $z_3 = \py{h}\py{d}\py{i}\py{e}i$. Berechnen Sie bitte die folgenden Ausdrücke
\setcounter{subenum}{0}
\vspace{0,2cm}\\
\setstretch{2}
\newitem{$\begin{aligned}[t]
	z_1 - 2z_2 + 3z_3
	\end{aligned}$ 
	\newline
	\ifnum\MitLoesungen=1 
	Richtige Antwort: \boxed{???}.
	\fi 
	\newline }
\newitem{$\begin{aligned}[t]
	2z_1 \cdot \bar{z}_2
	\end{aligned}$ }
\newitem{$\begin{aligned}[t]
	\frac{\bar{z}_1 \cdot \bar{z}_2}{z_3}
	\end{aligned}$ }
\newitem{$\begin{aligned}[t]
	\frac{z_1-\bar{z}_2}{3\bar{z}_3}
	\end{aligned}$ }
\setstretch{1}\\  

%%%%%%%%%%%%%%%%%%%%%%%%%%%%%%%%%
%%DGL
%%%%%%%%%%%%%%%%%%%%%%%%%%%%%%%%%
%%%%%%%%%%%%%%%%%%%%%%%%%%%%%%%%%
%%	AUFGABE 26
%%%%%%%%%%%%%%%%%%%%%%%%%%%%%%%%%
\begin{pycode}
from random import choice, randint
a = randint(2,5)
b = randint(2,4)
\end{pycode}

\Aufgabe{26}
Lösen Sie bitte die folgenden Differentialgleichungen durch Trennung der Variablen.
\begin{multicols}{2}
	\begin{enumerate}[(a)]
		\item $\begin{aligned}[t]
		\py{a}x^{\py{b}}y' = y^{\py{b}}
		%2x^2y' = y^2
		\end{aligned}$ 
		\newline
		\ifnum\MitLoesungen=1 
			Richtige Antwort: \boxed{???}.
		\fi 
		\newline
		\item $\begin{aligned}[t]
		y' = (y+2)^2
		\end{aligned}$ 
		\item $\begin{aligned}[t]
		y'(1+x^3) = 3x^2y
		\end{aligned}$ 
	\end{enumerate}
\end{multicols}

%%%%%%%%%%%%%%%%%%%%%%%%%%%%%%%%%
%%	AUFGABE 27
%%%%%%%%%%%%%%%%%%%%%%%%%%%%%%%%%
\begin{pycode}
from random import choice, randint
a = randint(2,5)
b = randint(2,4)
c = choice(['\sin','\cos','\ttan'])
\end{pycode}

\Aufgabe{27}
Lösen Sie bitte die folgenden Anfangswertprobleme.
\begin{multicols}{2}
	\begin{enumerate}[a)]
		\item $\begin{aligned}[t]
		y' + y\py{c}(x) = 0 \quad , y(\pi) = \frac{1}{e}
		%y' + y\sin(x) = 0 \quad , y(\pi) = \frac{1}{e}
		\end{aligned}$ 
		\newline
		\ifnum\MitLoesungen=1 
			Richtige Antwort: \boxed{???}.
		\fi 
		\newline
		\item $\begin{aligned}[t]
		y' + \frac{y}{x} = \frac{\ln(x)}{x} \quad  , y(1) = 1
		\end{aligned}$ 
		\item $\begin{aligned}[t]
		(x-1)(x+1)y' = y \quad  , y(2) = 1
		\end{aligned}$ 
		\item $\begin{aligned}[t]
		y' = 3x^2y + e^{x^3}\cos(x) \quad  , y(0) = 2
		\end{aligned}$ 
		%    \item $\begin{aligned}[t]
		%            y'(1+x^3) = 3x^2y
		%        	\end{aligned}$   	
	\end{enumerate}
\end{multicols}
\end{document}